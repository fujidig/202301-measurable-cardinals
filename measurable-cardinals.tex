\documentclass[uplatex]{jsarticle}
\usepackage[utf8]{inputenc}

\usepackage{amssymb}
\usepackage{amsmath}
\usepackage{amsthm}
\usepackage{framed}
\usepackage{braket}
\usepackage{bm}
\usepackage{mathrsfs}
\usepackage{accents}
\usepackage{tocloft}
\usepackage[dvipdfmx]{graphicx}
\usepackage{tikz}
\usepackage{url}
\usepackage{color}
\usepackage{xifthen}
\usepackage{xcolor}
\usepackage{framed}
\usepackage{mathtools}
\usepackage[explicit]{titlesec}
\usepackage{mdframed}
\usepackage{geometry}
\geometry{left=30mm,right=30mm,top=20mm,bottom=20mm}
\usepackage{enumerate}
\usepackage[dvipdfmx]{hyperref}
\renewcommand{\baselinestretch}{1.2}

\usetikzlibrary{positioning}
\usetikzlibrary{calc}
\usetikzlibrary{decorations.pathreplacing}
\usetikzlibrary{cd}


\renewcommand{\labelenumi}{(\arabic{enumi})}

\newcommand{\scrN}{\mathcal{N}}
\newcommand{\scrI}{\mathcal{I}}
\newcommand{\scrC}{\mathcal{C}}
\newcommand{\scrJ}{\mathcal{J}}
\newcommand{\N}{\mathbb{N}}
\newcommand{\Z}{\mathbb{Z}}
\renewcommand{\P}{\mathbb{P}}
\newcommand{\B}{\mathbb{B}}
\newcommand{\Q}{\mathbb{Q}}
\newcommand{\R}{\mathbb{R}}
\newcommand{\C}{\mathbb{C}}
\newcommand{\range}{\operatorname{ran}}
\newcommand{\dom}{\operatorname{dom}}
\newcommand{\append}{{}^\frown}
\newcommand{\boldsig}{\boldsymbol{\Sigma}}
\newcommand{\boldpi}{\boldsymbol{\Pi}}
\newcommand{\bolddelta}{\boldsymbol{\Delta}}
\newcommand{\Ordinals}{\mathrm{On}}
\newcommand\forces{\Vdash}
\newcommand\notforces{\nVdash}
\newcommand{\cl}{\operatorname{cl}}
\newcommand{\intr}{\operatorname{int}}
\newcommand{\ro}{\operatorname{ro}}
\newcommand{\rank}{\operatorname{rank}}
\newcommand{\frakt}{\mathfrak{t}}
\newcommand{\s}{\mathfrak{s}}
\newcommand{\frakb}{\mathfrak{b}}
\newcommand{\frakd}{\mathfrak{d}}
\newcommand{\frakc}{\mathfrak{c}}
\newcommand{\Pow}{\mathcal{P}}
\newcommand{\non}{\operatorname{non}}
\newcommand{\cov}{\operatorname{cov}}
\newcommand{\add}{\operatorname{add}}
\newcommand{\cof}{\operatorname{cof}}
\newcommand{\Cof}{\mathbf{Cof}}
\newcommand{\Cov}{\mathbf{Cov}}
\newcommand{\D}{\mathbf{D}}
\newcommand{\Lc}{\mathbf{Lc}}
\newcommand{\nul}{\mathsf{null}}
\newcommand{\meager}{\mathsf{meager}}
\newcommand{\id}{\mathrm{id}}
\newcommand{\diam}{\mathrm{diam}}
\newcommand{\height}{\mathrm{ht}}
\newcommand{\pow}{\mathrm{pow}}
\newcommand{\GTle}{\preceq_\mathrm{GT}}
\newcommand{\Map}[2]{\operatorname{Map}(#1, #2)}
\newcommand{\omegaupomega}{\omega^{\uparrow \omega}}
\newcommand{\twototheltomega}{2^{<\omega}}
\newcommand{\cf}{\operatorname{cf}}
\newcommand{\LangL}{\mathcal{L}}
\newcommand{\Add}{\operatorname{Add}}
\newcommand{\Seq}{\operatorname{Seq}}
\newcommand{\stem}{\operatorname{stem}}
\newcommand{\suc}{\operatorname{succ}}
\newcommand{\Lev}{\operatorname{Lev}}
\newcommand{\AND}{\mathbin{\&}}
\newcommand{\OR}{\text{ or }}
\newcommand{\restrict}{\upharpoonright}
\newcommand{\Lim}{\mathrm{Lim}}
\newcommand{\Limone}{\mathrm{Lim}_{\omega_1}}
\newcommand{\ZFC}{\mathsf{ZFC}}
\newcommand{\CH}{\mathsf{CH}}
\newcommand{\subsetic}{\subseteq_{\mathrm{ic}}}
\DeclareMathOperator*{\diagintr}{\triangle}

\newcommand{\seq}[1]{{\langle#1\rangle}}
\DeclarePairedDelimiter\abs{\lvert}{\rvert}
\DeclarePairedDelimiter\floor{\lfloor}{\rfloor}
\DeclarePairedDelimiter\ceil{\lceil}{\rceil}

\renewcommand\emptyset{\varnothing}
\renewcommand\subset{\subseteq}
\renewcommand{\setminus}{\smallsetminus}

\newcommand{\needtocheck}[1][]{%
	\ifthenelse{\equal{#1}{}}{%
		\textcolor{blue}{[NeedToCheck]}%
	}{%
		\textcolor{blue}{[NeedToCheck: #1]}%
	}%
}

\newcommand{\todo}[1][]{%
	\ifthenelse{\equal{#1}{}}{%
		\textcolor{red}{[TODO]}%
	}{%
		\textcolor{red}{[TODO: #1]}%
	}%
}


\theoremstyle{definition}
\newtheorem{thm}{定理}[section]
\newtheorem*{thm*}{定理}
\newtheorem{defi}[thm]{定義}
\newtheorem*{defi*}{定義}
\newtheorem{lem}[thm]{補題}
\newtheorem*{lem*}{補題}
\newtheorem{fact}[thm]{事実}
\newtheorem*{fact*}{事実}
\newtheorem{prop}[thm]{命題}
\newtheorem*{prop*}{命題}
\newtheorem{exm}[thm]{例}
\newtheorem*{exm*}{例}
\newtheorem{rmk}[thm]{注意}
\newtheorem*{rmk*}{注意}
\newtheorem{cor}[thm]{系}
\newtheorem*{cor*}{系}
\newtheorem*{notation*}{記法}
\newtheorem{asm}[thm]{仮定}
\newtheorem{prob}[thm]{問題}
\newtheorem{conj}[thm]{予想}
\renewcommand{\proofname}{証明}

\newenvironment{claim}[1]{\par\noindent\underline{主張 #1:}\space}{}
\newenvironment{claimproof}[1]{\par\noindent$\because$) \space#1}{\hfill //}

\usepackage[backend=biber,style=alphabetic,sorting=nty,doi=false,isbn=false,url=false,eprint=true]{biblatex}
%\addbibresource{oracle-cc.bib}
\renewbibmacro{in:}{}

\title{\Huge 可測基数ノート}
\author{でぃぐ}

\begin{document}
	\maketitle
	
	%\begin{abstract}
	%	hogehoge
	%end{abstract}
	
	\tableofcontents
	
	
	\section{可測基数の初歩}
	
	\begin{defi}
		\begin{enumerate}
			\item 基数$\kappa$が\textbf{可測基数}であるとは,$\kappa$上の$\kappa$-完備な非単項超フィルターが存在することを言う.
			\item 基数$\kappa$が\textbf{実数値可測基数}であるとは,$\kappa$上の非自明な$\kappa$完備測度が存在することを言う.
		\end{enumerate}
	\end{defi}

	\begin{lem}
		$\kappa$を次を満たす\.最\.小\.の基数とする:非単項$\sigma$-完備な超フィルターが存在する.
		$U$をそのような超フィルターの一つとする.
		このとき,$U$は$\kappa$-完備である.
	\end{lem}
	\begin{proof}
		$U$が$\kappa$-完備でないと仮定する.
		すると$\kappa$の分割$\{ X_\alpha : \alpha < \gamma \}$があって,$\gamma < \kappa$かつ各$X_\alpha$は$U$の意味で小さい.
		$f \colon \kappa \to \gamma$を次で定める:
		\[
			f(x) = \alpha \iff x \in X_\alpha.
		\]
		つまり,各入力$x < \kappa$について,$x$が何番目のピースに属しているかを返す関数である.
		$\gamma$上の超フィルター$D$を
		\[
			D = \{ Z \subset \gamma : f^{-1}(Z) \in U \}
		\]
		で定める.$U$が$\sigma$完備なので,$D$も$\sigma$完備である.
		$D$は非単項でもある:なぜなら,各$\alpha < \gamma$について$f^{-1}\{\alpha\} = X_\alpha \not \in U$より$\alpha \not \in D$だからである.
		したがって,$D$は$\gamma$上の単項$\sigma$-完備な超フィルターだが,$\gamma < \kappa$より,これは$\kappa$の最小性に矛盾.
	\end{proof}

	\begin{lem}
	可測基数は到達不能基数である.
	\end{lem}
	\begin{proof}
		$\kappa$を可測基数とする.
		
		$\kappa$の正則性を示す.
		$\kappa$上の$\kappa$-完備な非単項超フィルター$U$を取る.
		$\kappa$が特異だとすると,$\kappa$の共終列$\seq{\lambda_i : i < \cf(\kappa) }$でおのおのの$\lambda_i$は$\kappa$未満なものが取れる.
		今,$\kappa = \bigcup_{i < \cf(\kappa)} \lambda_i$である.
		左辺$\kappa$は$U$に属するが,右辺はおのおのの$\lambda_i$が$U$の意味で小さく,その$\cf(\kappa) < \kappa$個の和集合だから$U$の意味で小さい.矛盾した.
		なお,ここで,おのおのの$\lambda_i$が小さいのは各1点集合が小さく,$\lambda_i$はその$\lambda_i < \kappa$個の和集合として書けるからである.
		
		$\kappa$の強極限性を示す.背理法で,ある$\lambda < \kappa$について,$2^\lambda \ge \kappa$だと仮定する.
		集合$S \subset \{0, 1\}^\lambda$で$\abs{S} = \kappa$となるものを取る.
		集合$S$上の$\kappa$-完備な非単項超フィルター$U$を取る.
		各$\alpha \in \lambda$について集合$X_\alpha \subset S$を
		\[
		\{ f \in S : f(\alpha) = 0 \} \text{ もしくは } \{ f \in S : f(\alpha) = 1 \}
		\]
		で$U$に属する方とする.集合$X$を
		\[
		X = \bigcap_{\alpha < \lambda} X_\alpha
		\]
		で定めると$X \in U$であるが,明らかに$X$は1点集合である.これは$U$の非単項性に矛盾.
	\end{proof}

	\section{Ulamの定理の証明}
	
	\section{正規フィルター}
	
	\section{可測基数の存在と実数値可測基数の存在の無矛盾等価性}
	
	\section{ジェネリック超冪}
	
	\nocite{*}
	\printbibliography[title={参考文献}]
	
\end{document}
