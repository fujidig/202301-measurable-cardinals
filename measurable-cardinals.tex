\documentclass[uplatex]{jsarticle}
\usepackage[utf8]{inputenc}

\usepackage{amssymb}
\usepackage{amsmath}
\usepackage{amsthm}
\usepackage{framed}
\usepackage{braket}
\usepackage{bm}
\usepackage{mathrsfs}
\usepackage{accents}
\usepackage{tocloft}
\usepackage[dvipdfmx]{graphicx}
\usepackage{tikz}
\usepackage{url}
\usepackage{color}
\usepackage{xifthen}
\usepackage{xcolor}
\usepackage{framed}
\usepackage{mathtools}
\usepackage[explicit]{titlesec}
\usepackage{mdframed}
\usepackage{geometry}
\geometry{left=30mm,right=30mm,top=20mm,bottom=20mm}
\usepackage{enumerate}
\usepackage[dvipdfmx]{hyperref}
\usepackage{pxjahyper}
\renewcommand{\baselinestretch}{1.1}

\usetikzlibrary{positioning}
\usetikzlibrary{calc}
\usetikzlibrary{decorations.pathreplacing}
\usetikzlibrary{cd}


\renewcommand{\labelenumi}{(\arabic{enumi})}

\newcommand{\scrN}{\mathcal{N}}
\newcommand{\scrI}{\mathcal{I}}
\newcommand{\scrC}{\mathcal{C}}
\newcommand{\scrJ}{\mathcal{J}}
\newcommand{\N}{\mathbb{N}}
\newcommand{\Z}{\mathbb{Z}}
\renewcommand{\P}{\mathbb{P}}
\newcommand{\B}{\mathbb{B}}
\newcommand{\Q}{\mathbb{Q}}
\newcommand{\R}{\mathbb{R}}
\newcommand{\C}{\mathbb{C}}
\newcommand{\range}{\operatorname{ran}}
\newcommand{\dom}{\operatorname{dom}}
\newcommand{\append}{{}^\frown}
\newcommand{\boldsig}{\boldsymbol{\Sigma}}
\newcommand{\boldpi}{\boldsymbol{\Pi}}
\newcommand{\bolddelta}{\boldsymbol{\Delta}}
\newcommand{\Ordinals}{\mathrm{On}}
\newcommand\forces{\Vdash}
\newcommand\notforces{\nVdash}
\newcommand{\cl}{\operatorname{cl}}
\newcommand{\intr}{\operatorname{int}}
\newcommand{\ro}{\operatorname{ro}}
\newcommand{\rank}{\operatorname{rank}}
\newcommand{\frakt}{\mathfrak{t}}
\newcommand{\s}{\mathfrak{s}}
\newcommand{\frakb}{\mathfrak{b}}
\newcommand{\frakd}{\mathfrak{d}}
\newcommand{\frakc}{\mathfrak{c}}
\newcommand{\Pow}{\mathcal{P}}
\newcommand{\non}{\operatorname{non}}
\newcommand{\cov}{\operatorname{cov}}
\newcommand{\add}{\operatorname{add}}
\newcommand{\cof}{\operatorname{cof}}
\newcommand{\Cof}{\mathbf{Cof}}
\newcommand{\Cov}{\mathbf{Cov}}
\newcommand{\D}{\mathbf{D}}
\newcommand{\Lc}{\mathbf{Lc}}
\newcommand{\nul}{\mathsf{null}}
\newcommand{\meager}{\mathsf{meager}}
\newcommand{\id}{\mathrm{id}}
\newcommand{\diam}{\mathrm{diam}}
\newcommand{\height}{\mathrm{ht}}
\newcommand{\pow}{\mathrm{pow}}
\newcommand{\GTle}{\preceq_\mathrm{GT}}
\newcommand{\Map}[2]{\operatorname{Map}(#1, #2)}
\newcommand{\omegaupomega}{\omega^{\uparrow \omega}}
\newcommand{\twototheltomega}{2^{<\omega}}
\newcommand{\cf}{\operatorname{cf}}
\newcommand{\LangL}{\mathcal{L}}
\newcommand{\Add}{\operatorname{Add}}
\newcommand{\Seq}{\operatorname{Seq}}
\newcommand{\stem}{\operatorname{stem}}
\newcommand{\suc}{\operatorname{succ}}
\newcommand{\Lev}{\operatorname{Lev}}
\newcommand{\AND}{\mathbin{\&}}
\newcommand{\OR}{\text{ or }}
\newcommand{\restrict}{\upharpoonright}
\newcommand{\Lim}{\mathrm{Lim}}
\newcommand{\Limone}{\mathrm{Lim}_{\omega_1}}
\newcommand{\ZFC}{\mathsf{ZFC}}
\newcommand{\CH}{\mathsf{CH}}
\newcommand{\subsetic}{\subseteq_{\mathrm{ic}}}
\newcommand{\crit}{\operatorname{crit}}
\newcommand{\Ult}{\operatorname{Ult}}
\newcommand{\ext}{\operatorname{ext}}
\newcommand{\statone}{\mathsf{stat}_{\omega_1}}
\DeclareMathOperator*{\diagintr}{\triangle}

\newcommand{\seq}[1]{{\langle#1\rangle}}
\DeclarePairedDelimiter\abs{\lvert}{\rvert}
\DeclarePairedDelimiter\floor{\lfloor}{\rfloor}
\DeclarePairedDelimiter\ceil{\lceil}{\rceil}

\renewcommand\emptyset{\varnothing}
\renewcommand\subset{\subseteq}
\renewcommand{\setminus}{\smallsetminus}

\newcommand{\needtocheck}[1][]{%
	\ifthenelse{\equal{#1}{}}{%
		\textcolor{blue}{[NeedToCheck]}%
	}{%
		\textcolor{blue}{[NeedToCheck: #1]}%
	}%
}

\newcommand{\todo}[1][]{%
	\ifthenelse{\equal{#1}{}}{%
		\textcolor{red}{[TODO]}%
	}{%
		\textcolor{red}{[TODO: #1]}%
	}%
}


\theoremstyle{definition}
\newtheorem{thm}{定理}[section]
\newtheorem*{thm*}{定理}
\newtheorem{defi}[thm]{定義}
\newtheorem*{defi*}{定義}
\newtheorem{lem}[thm]{補題}
\newtheorem*{lem*}{補題}
\newtheorem{fact}[thm]{事実}
\newtheorem*{fact*}{事実}
\newtheorem{prop}[thm]{命題}
\newtheorem*{prop*}{命題}
\newtheorem{exm}[thm]{例}
\newtheorem*{exm*}{例}
\newtheorem{rmk}[thm]{注意}
\newtheorem*{rmk*}{注意}
\newtheorem{cor}[thm]{系}
\newtheorem*{cor*}{系}
\newtheorem*{notation*}{記法}
\newtheorem{asm}[thm]{仮定}
\newtheorem{prob}[thm]{演習問題}
\newtheorem{conj}[thm]{予想}
\newtheorem{defiandlem}[thm]{定義と補題}
\renewcommand{\proofname}{証明}

\newenvironment{claim}[1]{\par\noindent\underline{主張 #1:}\space}{}
\newenvironment{claimproof}[1]{\par\noindent$\because$) \space#1}{\hfill //}

\usepackage[backend=biber,style=alphabetic,sorting=nty,doi=false,isbn=false,url=false,eprint=true]{biblatex}
\addbibresource{measurable-cardinals.bib}
\renewbibmacro{in:}{}


\usepackage{titling}
\renewcommand\maketitlehooka{
	\vspace{-1.5cm}
	\noindent\vrule height 2.5pt width \textwidth
}
\pretitle{
	\begin{center}
		\Huge\bfseries
		\vspace{-0.5cm}
	}
	\posttitle{
	\end{center}
}
\preauthor{
	\begin{flushright}
		\large
		\vspace{-0.5cm}
	}
	\postauthor{
	\end{flushright}
}
\predate{
	\begin{flushright}
		\large
		\vspace{-0.5cm}
	}
	\postdate{
	\end{flushright}
	\vspace{-0.5cm}\noindent\vrule height 2.5pt width \textwidth
}

\title{可測基数ノート}
\author{でぃぐ}

\begin{document}
	
	\maketitle
	
	\begin{abstract}
		本稿は可測基数についてのノートである.可測基数のかなり初歩的な話からはじめ,超冪と初等埋め込みという標準的な話題を扱い,最後に応用として峻厳イデアルの存在の無矛盾性証明を行う.
	\end{abstract}
	
	\tableofcontents
	
	\vspace{1cm}
	
	本稿の内容はほぼJechのテキスト\cite{jech2006set}を参考にしている.
	
	\section{可測基数の初歩}
	
	可測基数の研究は,Lebesgue測度を$\R$の冪集合全体に拡張できるかという問から出発している.本節ではその命題が$\ZFC$の無矛盾性を超えることを示す.
	
	\begin{defi}
		$S$を無限集合とする. $S$上の(一様かつ$\sigma$加法的な確率)\textbf{測度}とは$\mu \colon \Pow(S) \to [0, 1]$であって,次を満たすものである:
		\begin{enumerate}
			\item $\mu(\emptyset) = 0, \mu(S) = 1$.
			\item $X \subset Y \subset S$なら,$\mu(X) \le \mu(Y)$.
			\item (一様性) 任意の$s \in S$について$\mu(\{s\}) = 0$.
			\item ($\sigma$加法性) $X_n, n \in \omega$が互いに素な$S$の部分集合たちであれば,
			\[
			\mu\left(\bigcup_{n \in \omega} X_n\right) = \sum_{n \in \omega} \mu(X_n). 
			\]
		\end{enumerate}
	\end{defi}
	
	測度論で扱う測度は$S$上のある$\sigma$加法族を定義域とするものであったが,ここで扱う測度は定義域が$\Pow(S)$なことに注意しよう.
	
	\begin{defi}
		$\mu$を$S$上の測度とする.$A \subset S$が\textbf{原子}であるとは,$\mu(A) > 0$かつ任意の$X \subset A$に対して$\mu(X) = 0$または$\mu(X) = \mu(A)$となるものである.
		原子が存在しない測度を原子なしの測度という.
	\end{defi}
	
	\begin{defi}
		\begin{enumerate}
			\item 基数$\kappa$が\textbf{可測基数}であるとは,$\kappa$上の$\kappa$完備な非単項超フィルターが存在することを言う.
			\item 基数$\kappa$が\textbf{実数値可測基数}であるとは,$\kappa$上の$\kappa$加法的測度が存在することを言う.
		\end{enumerate}
	\end{defi}
	
	$S$上の非単項超フィルターを考えることと,$S$上の値域が$\{0, 1\}$である(つまり,2値である)測度を考えることは同じである.
	
	実際,非単項超フィルター$U$に対して
	\[\mu(X) = \begin{cases}1 & (X \in U)\\ 0 & (X \not \in U) \end{cases}\]
	で定義される測度を対応される写像と,2値測度$\mu$に対して非単項超フィルター
	\[U = \mu^{-1}\{1\}\]
	を対応させる写像は互いの逆写像である.
	
	また,この対応において,超フィルターが$\kappa$完備なことと測度が$\kappa$加法的なことが対応する.
	よって,可測基数は実数値可測基数である.
	
	\begin{defi}
		集合$S$上のイデアル$I$で\textbf{$\sigma$飽和的}であるとは,$I$に属さない$S$の部分集合族で互いに素なものはどれも,族の濃度が可算であることを意味する.
	\end{defi}
	
	$S$上の測度$\mu$から来るイデアル$I_\mu = \mu^{-1}\{0\}$は必ず$\sigma$飽和的である.
	なぜなら,$\mathcal{A}$が$I$に属さない (すなわち$\mu$の測度が正な)部分集合の族で互いに素なものとしよう.
	このとき正の自然数$n$に対して$\mu(A) > 1/n$を満たす$A \in \mathcal{A}$は$n$個しかない.
	よって,$\mathcal{A}$は有限集合の可算和であるから,たかだか可算濃度を持つ.
	
	\begin{lem}\label{lem:measisreg}
		実数値可測基数 (および可測基数)は正則基数である.
	\end{lem}
	\begin{proof}
		$\kappa$を実数値可測基数とする.
		$\kappa$上の$\kappa$完備な測度$\mu$を取る.
		$\kappa$が特異だとすると,$\kappa$の共終列$\seq{\lambda_i : i < \cf(\kappa) }$でおのおのの$\lambda_i$は$\kappa$未満なものが取れる.
		今,$\kappa = \bigcup_{i < \cf(\kappa)} \lambda_i$である.
		左辺$\kappa$は測度$1$だが,右辺はおのおのの$\lambda_i$が測度$0$で,その$\cf(\kappa) < \kappa$個の和集合だから測度$0$である.矛盾した.
		なお,ここで,おのおのの$\lambda_i$が測度$0$なのは,各1点集合が測度$0$で,$\lambda_i$はその$\lambda_i < \kappa$個の和集合として書けるからである.
	\end{proof}

	\begin{lem}\label{lem:measisinacc}
		可測基数は到達不能基数である.
	\end{lem}
	\begin{proof}
		$\kappa$を可測基数とする.
		
		$\kappa$が正則なことは補題\ref{lem:measisreg}で示した.
		
		$\kappa$の強極限性を示す.背理法で,ある$\lambda < \kappa$について,$2^\lambda \ge \kappa$だと仮定する.
		集合$S \subset \{0, 1\}^\lambda$で$\abs{S} = \kappa$となるものを取る.
		集合$S$上の$\kappa$完備な非単項超フィルター$U$を取る.
		各$\alpha \in \lambda$について集合$X_\alpha \subset S$を
		\[
		\{ f \in S : f(\alpha) = 0 \} \text{ もしくは } \{ f \in S : f(\alpha) = 1 \}
		\]
		で$U$に属する方とする.集合$X$を
		\[
		X = \bigcap_{\alpha < \lambda} X_\alpha
		\]
		で定めると$X \in U$であるが,明らかに$X$は1点集合である.これは$U$の非単項性に矛盾.
	\end{proof}
	
	\begin{lem}
		\begin{enumerate}
			\item $\kappa$を次を満たす最小の基数とする:非単項$\sigma$完備な超フィルターが存在する.
			$U$をそのような超フィルターの一つとする.
			このとき,$U$は$\kappa$完備である.
			\item $\kappa$を次を満たす最小の基数とする: $\kappa$上の測度が存在する.
			$\mu$をそのような測度とする.
			このとき測度$0$集合のイデアル$I_\mu$は$\kappa$完備である.
			\item $\kappa$を次を満たす最小の基数とする: $\kappa$上の$\sigma$完備かつ $\sigma$飽和的イデアルが存在する.
			$I$をそのようなイデアルとする.
			このとき$I$は$\kappa$完備である.
		\end{enumerate}
	\end{lem}
	\begin{proof}
		(1). 
		$U$が$\kappa$完備でないと仮定する.
		すると$\kappa$の分割$\{ X_\alpha : \alpha < \gamma \}$があって,$\gamma < \kappa$かつ各$X_\alpha$は$U$の意味で小さい.
		関数$f \colon \kappa \to \gamma$を次で定める:
		\[
		f(x) = \alpha \iff x \in X_\alpha.
		\]
		つまり,各入力$x < \kappa$について,$x$が何番目のピースに属しているかを返す関数である.
		$\gamma$上の超フィルター$D$を
		\[
		D = \{ Z \subset \gamma : f^{-1}(Z) \in U \}
		\]
		で定める.$U$が$\sigma$完備なので,$D$も$\sigma$完備である.
		$D$は非単項でもある:なぜなら,各$\alpha < \gamma$について$f^{-1}\{\alpha\} = X_\alpha \not \in U$より$\alpha \not \in D$だからである.
		したがって,$D$は$\gamma$上の単項$\sigma$-完備な超フィルターだが,$\gamma < \kappa$より,これは$\kappa$の最小性に矛盾.
		
		(2). $I_\mu$が$\kappa$完備ではないと仮定する.
		すると測度$0$集合の族$\{X_\alpha : \alpha < \gamma \}$で,$\gamma < \kappa$かつ,それらの和集合$X = \bigcup_{\alpha < \gamma} X_\alpha$は測度正なものがとれる.
		$X_\alpha$たちは互いに素であると仮定しても良い.
		$f \colon X \to \gamma$を
		\[
		f(x) = \alpha \iff x \in X_\alpha
		\]
		と定め,$\gamma$上の測度$\nu$を
		\[
		\nu(Z) = \frac{\mu(f^{-1}(Z))}{\mu(X)} 
		\]
		と定める.
		$\nu$は$\sigma$加法的である.
		また,$\nu$は一様である,なぜなら,各$\alpha < \gamma$について$\nu(\{\alpha\}) = \frac{\mu(X_\alpha)}{\mu(X)} = 0$だからである.
		これは$\kappa$の最小性に反する.
		
		(3)の証明は(1)や(2)と同様である.
	\end{proof}
	
	$\mu$を集合$S$上の測度とし,$I_\mu$を測度$0$集合のイデアルとすれば,$\mu$が$\kappa$加法的なら,$I_\mu$が$\kappa$完備なことは明らかである.逆も言える:
	
	\begin{lem}
		$\mu$を集合 $S$上の測度とし,$I_\mu$を測度$0$集合のイデアルとする.
		このとき,もし$I_\mu$が$\kappa$完備なら,$\mu$は$\kappa$加法的である.
	\end{lem}
	\begin{proof}
		$\gamma < \kappa$とし,$\seq{X_\alpha : \alpha < \gamma}$を互いに素な$S$の部分集合の族とする.
		$X_\alpha$たちが互いに素なので,そのうちたかだか可算個が正の測度を持つ.
		よって,
		\[
		\{ X_\alpha : \alpha < \gamma \} = \{ Y_n : n \in \omega \} \cup \{ Z_\alpha : \alpha < \gamma \}
		\]
		と書くことができる.ここに各$Z_\alpha$は測度$0$集合.よって,
		\[
		\mu(\bigcup_{\alpha < \gamma} X_\alpha) = \mu(\bigcup_{n \in \omega} Y_n) + \mu(\bigcup_{\alpha < \gamma} Z_\alpha)
		\]
		を得る.$\mu$が$\sigma$加法的なので,
		\[
		\mu(\bigcup_{n \in \omega} Y_n) =  \sum_{n \in \omega} \mu(Y_n) 
		\]
		である.また,$I_\mu$が$\kappa$完備なので,
		\[
		\mu(\bigcup_{\alpha < \gamma} Z_\alpha) = 0
		\]
		である.以上より,
		\[
		\mu(\bigcup_{\alpha < \gamma} X_\alpha) = \sum_{\alpha < \gamma} \mu(X_\alpha)
		\]
		を得る.
	\end{proof}
	
	\begin{lem}\label{lem:continuum}
		\begin{enumerate}
			\item ある集合上の原子なしの測度が存在するとき,ある基数$\kappa \le 2^{\aleph_0}$上に測度が存在する.
			\item $I$を集合$S$上の$\sigma$完備$\sigma$飽和的イデアルとする.このとき,ある$Z \subset S$に対して$I \restrict Z = \{ X \subset Z : X \in I \}$が極大イデアルであるか,または,$\sigma$完備$\sigma$飽和的イデアルがある$\kappa \le 2^{\aleph_0}$上に存在するかのどちらかが成り立つ.
		\end{enumerate}
	\end{lem}
	\begin{proof}
		(1). $\mu$をそのような測度とする.$S$の測度正な部分集合からなり,逆向きの包含関係で順序付けられた木$T$を構成する.
		$T$の根は$S$である.
		各$X \in T$について,$X$の測度正な集合への分割$X = Y \cup Z, Y \cap Z = \emptyset$を取り,この2つを$X$の直後の元とする.
		$\alpha$が極限順序数のとき$T$の第$\alpha$レベルにはすべての共通部分$X = \bigcap_{\xi < \alpha} X_\xi$であって,$\seq{X_\xi : \xi < \alpha}$は$T \restrict \alpha$の増大鎖で$X_\xi$は第$\xi$レベルの元,$X$は測度正なものたちを置く.
		
		$T$のどの枝も可算である:なぜなら,$\seq{X_\xi : \xi < \alpha}$が枝ならば,$\seq{ X_\xi \setminus X_{\xi+1} : \xi < \alpha }$は測度正な集合の互いに素な族となるからである.
		
		同様に,$T$のどのレベルも可算であることも分かる.
		よって,$T$はたかだか$2^{\aleph_0}$個の極大枝を持つ (各$\alpha < \omega_1$について高さ$\alpha$の極大枝の個数はたかだか$2^{\aleph_0}$.よってそれらの$\omega_1$個の和集合でたかだか$2^{\aleph_0}$個となる).
		
		$\{ b_\alpha : \alpha < \kappa \}$, $\kappa \le 2^{\aleph_0}$をすべての極大枝$b = \{ X_\xi : \xi < \gamma \}$であって,$\bigcap_{\xi < \gamma} X_\xi$が非空なものの枚挙とする.
		各$\alpha < \kappa$について$Z_\alpha = \bigcap b_\alpha$とおく.
		$\{ Z_\alpha : \alpha < \kappa \}$は$S$の測度$0$集合への分割となる 
		($Z_\alpha$が測度$0$でないとすると,一個高さを上げることができ枝の極大性に反する; また,互いの異なる極大枝$b_\alpha$と$b_\beta$はどこかで枝分かれしているはずだから,後続ステップでの構成の仕方より,$X_\alpha \cap X_\beta = \emptyset$を得る; 覆っていることは$s \in S$を任意に取るとき,$s$が入っている集合を根から追跡することにより,ある$X_\alpha$に$s$が入っていることがわかるからよい).
		あとは$f \colon S \to \kappa$を$f(x) = \alpha \iff x \in Z_\alpha$とおき,$\kappa$上の測度$\nu$を$\nu(Z) = \mu(f^{-1}(Z))$とおけば,$\nu$は一様な$\sigma$加法的測度である.
		
		%$\nu$が原子なしであることを示そう.
		%$Z$が$\nu$の原子であると仮定する.
		%$Y = f^{-1}(Z)$とおく.
		
		%もし$X \in T$かつ$\mu(X \cap Y) \ne 0$であって,$X_1, X_2$を$X$の$X$の2つの直後の元とするとき,$\mu(X_1 \cap Y) = 0$または$\mu(X_2 \cap Y) = 0$である.なぜなら,両方とも正だとすると,$f$の定め方より$f(X_1)$と$f(X_2)$は互いに素であって,どちらの$i < 2$についても$\nu(f(X_i) \cap Z) = \mu(f^{-1}(f(X_i) \cap Z)) > \mu(X_i \cap Y) > 0$となり,$Z$が原子であることに反するからである.
		
		%このことと帰納法により,$T$の各レベルにおいて一意に$X$があって,$\mu(Y \cap X) \ne 0$を満たす.
		
		(2). (1)と同様である.
	\end{proof}
	
	\begin{cor}
		$\kappa$が実数値可測基数ならば,$\kappa$は可測基数か,$\kappa \le 2^{\aleph_0}$である.
		より一般に,$\kappa$が$\kappa$完備$\sigma$飽和的イデアルを持つと,$\kappa$は可測基数であるか,$\kappa \le 2^{\aleph_0}$である.
	\end{cor}
	\begin{proof}
		補題\ref{lem:continuum}の証明より,$\mu$が$S$上の原子なしの測度なら,$S$のたかだか$2^{\aleph_0}$個への測度$0$個の分割が存在することがわかる.つまり,$\mu$は$(2^{\aleph_0})^+$加法的ではない.
		したがって,原子なしの$\kappa$加法的測度を$\kappa$が持つとき,$\kappa \le 2^{\aleph_0}$である (結論の否定を取ると,$\kappa \ge (2^{\aleph_0})^+$だが,これと$\kappa$加法性より$(2^{\aleph_0})^+$加法性が出るから).
		後半の主張も同様.
	\end{proof}
	
	補題\ref{lem:continuum}の(1)の主張の結論には「原子なし」が含まれていなかったが,これは実際には「原子なし」と結論付けられる.なぜなら,原子があると$\kappa$は可測基数となるが,補題\ref{lem:measisinacc}より,それは$\kappa \le 2^{\aleph_0}$と相容れないからだ.
	
	\begin{defi}
		$(\aleph_1, \aleph_0)$-Ulam行列とは,$\omega_1$の部分集合の族$\seq{ A_{\alpha, n} : \alpha \in \omega_1, n \in \omega }$であって,次の2条件を満たすものである.
		\begin{enumerate}
			\item 各$n \in \omega$と異なる$\alpha, \beta \in \omega_1$について$A_{\alpha,n} \cap A_{\beta,n} = \emptyset$である.
			\item 各$\alpha \in \omega_1$について,集合$\omega_1 \setminus \bigcup_{n \in \omega} A_{\alpha,n}$はたかだか可算集合である.
		\end{enumerate}
	\end{defi}
	
	\[
	\tikz[scale=0.8]{
		\draw[->] (0, 0) -> (5, 0) node[right] {$\omega$};
		\draw[->] (0, 0) -> (0, -7.5) node[below] {$\omega_1$};
		\foreach \x in {1, 2, 3, 4, 5, 7, 8} {
			\draw (\x*0.5,-3) circle (0.2);
		}
		\foreach \y in {1, 2, ..., 14} {
			\draw (3, -\y*0.5) circle (0.2);
		}
		\node at (4.3, -3) [right] {$\cdots$ $\omega_1$をほとんど被覆};
		\node at (3, -7.5) {$\vdots$};
		\node at (3, -8) {互いに素};
		\draw (-0.2, -3) node[left] {$\alpha$} -- (0.2, -3);
		\draw (3, -0.2) node[above=0.4cm] {$n$} -- (3, 0.2);
		\draw [bend right = 20] (3+0.707*0.2,-3+0.707*0.2) to (3.7,-2) node[above] {$A_{\alpha,n}$};
	}
	\]
	
	\begin{lem}
		$(\aleph_1, \aleph_0)$-Ulam行列は存在する.
	\end{lem}
	\begin{proof}
		各$\xi \in \omega_1$に対して$f_\xi \colon \omega \to \omega_1$を$\xi \subset \range(f_\xi)$なるものとする.
		集合 $A_{\alpha,n}$を
		\[
		\xi \in A_{\alpha,n} \iff f_\xi(n) = \alpha
		\]
		と定める.
		
		$\xi \in A_{\alpha,n} \cap A_{\beta,n}$なら$\alpha = f_\xi(n) = \beta$となるので,Ulam行列の条件(1)が成り立っていることがわかる.
		
		$\alpha \in \omega_1$とする.
		$\xi > \alpha$に対して,$f_\xi$の取り方より,$f_\xi(n) = \alpha$となる$n \in \omega$が存在する.
		よって,
		\[
		[\alpha+1, \omega_1) \subseteq \bigcup_{n \in \omega} A_{\alpha,n}
		\]
		なので条件(2)も成り立っている.
	\end{proof}
	
	\begin{prob}
		$(\aleph_1, \aleph_0)$-Ulam行列の定義において,「各行は可算集合を除いてほとんど$\omega_1$を覆っている」という条件を「各行は$\omega_1$を(完全に)覆っている」と変更したバージョンは存在しないことを示せ.
	\end{prob}
	
	\begin{lem}
		$\omega_1$上の$\sigma$完備$\sigma$飽和的イデアルは存在しない.
		特に$\omega_1$上の測度は存在しない.
	\end{lem}
	\begin{proof}
		そのようなイデアル$I$が存在したと仮定する.
		また,$\seq{ A_{\alpha, n} : \alpha \in \omega_1, n \in \omega }$を$(\aleph_1, \aleph_0)$-Ulam行列とする.
		$I$の$\sigma$完備性とUlam行列の条件(2)より,各$\alpha$について自然数$n_\alpha$があって,$A_{\alpha,n}$は$I$-正である.
		したがって,鳩の巣原理より,$W \subset \omega_1$, $\abs{W} = \aleph_1$, $n \in \omega$があって,すべての$\alpha \in W$について$n_\alpha = n$である.
		すると$\{ A_{\alpha,n} : \alpha \in W \}$は互いに素 (by Ulam行列の条件(1))な非可算な$I$-正集合の族となる.これは$I$の$\sigma$飽和性に矛盾する.
	\end{proof}
	
	以上の$\omega_1$を一般の後続基数に一般化できる.証明は同様なので省略する.
	
	\begin{defiandlem}\label{dl:lambdaplusulam}
		$\lambda$を基数とする.
		\begin{enumerate}
			\item $(\lambda^+, \lambda)$-Ulam行列とは,$\lambda^+$の部分集合の族$\seq{ A_{\alpha, \eta} : \alpha \in \lambda^+, \eta \in \lambda }$であって,次の2条件を満たすものである.
			\begin{enumerate}
				\item 各$\eta \in \lambda$と異なる$\alpha, \beta \in \lambda^+$について$A_{\alpha,\eta} \cap A_{\beta,\eta} = \emptyset$である.
				\item 各$\alpha \in \lambda^+$について,集合$\lambda^+ \setminus \bigcup_{\eta \in \lambda} A_{\alpha,\eta}$は$\lambda$以下の濃度を持つ.
			\end{enumerate}
			\item $(\lambda^+, \lambda)$-Ulam行列は存在する.
			\item $\lambda^+$上の$\lambda^+$完備$\sigma$飽和的イデアルは存在しない.
		\end{enumerate}
	\end{defiandlem}
	
	\begin{cor}
		任意の実数値可測基数は,弱到達不能基数である.
	\end{cor}
	\begin{proof}
		$\kappa$を実数値可測基数とする.
		正則なことは補題\ref{lem:measisreg}で示した.
		後続基数でないことは,定義と補題\ref{dl:lambdaplusulam}から分かる.
	\end{proof}
	
	以上より次が結論付けられる:$\ZFC$に「ある集合上の測度が存在する」という命題を加えた公理系の無矛盾性の強さは$\ZFC$より真に強い.なぜなら「ある集合上の測度が存在する」からはその測度が原子ありかなしかに応じて,到達不能基数か弱到達不能基数のどちらかが出て,どちらも$\ZFC$の無矛盾性を出すからである.これがUlamが証明した定理である.
		
	\section{正規フィルター}
	
	フィルターが\textbf{正規}であるとは,それが対角共通部分を取る操作で閉じていることであった.
	また,$\kappa$上の$\kappa$完備な超フィルター$U$に対しては,$U$が正規であることと任意の押し下げ関数$f \colon X \to \kappa$, $X \in U$に対して,ある$Y \in U$について $f$が$Y$上で定数関数となることと同値であった.
	
	\begin{thm}
		任意の可測基数の上に正規超フィルターが存在する.
	\end{thm}
	\begin{proof}
		$U$を$\kappa$上の非単項$\kappa$完備超フィルターとする.$f, g \in \kappa^\kappa$に対して,
		\[
		f =^* g \iff \{ \alpha < \kappa : f(\alpha) = g(\alpha)\} \in U
		\]
		という同値関係を入れる.また,
		\[
		f <^* g \iff \{ \alpha < \kappa : f(\alpha) < g(\alpha)\} \in U
		\]
		という擬全順序関係を入れる.
		
		無限下降列$f_0 >^* f_1 >^* f_2 >^* \dots$は存在しない.実際,それがあれば$X_n = \{ \alpha : f_n(\alpha) > f_{n+1}(\alpha))\} \in U$だが,$U$が$\sigma$完備なので,$X = \bigcap_{n \in \omega} X_n \in U$であり,特に$X$は空でない.$\alpha \in X$を一つ取ると,順序数の無限下降列$f_0(\alpha) > f_1(\alpha) > f_2(\alpha) > \dots$ができて矛盾である.
		
		したがって,$<^*$は擬整列順序である.
		
		$f \colon \kappa \to \kappa$を次を満たす(この擬整列順序で)最小の関数とする:
		任意の$\gamma < \kappa$に対して,$\{\alpha : f(\alpha) > \gamma \} \in U$である.
		このような$f$は少なくとも1つ存在する.たとえば対角関数$d(\alpha) = \alpha$は条件を満たす.
		
		$D = f(U) = \{ X \subset \kappa : f^{-1}(X) \in U \}$とおく.
		$D$が$\kappa$上の正規超フィルターなことを示そう.
		
		各$\gamma < \kappa$に対して,$f^{-1}\{\gamma\} \not \in U$である ($f^{-1}[\gamma+1, \kappa) \in U$だから).よって,$\gamma \not \in D$なので,$D$は非単項である.
		
		$D$の正規性を示そう.
		$h$を$X \in D$上の押し下げ関数とする.
		$h$が$D$のあるメンバー上で定数なことを示さなければいけない.
		$g \in \kappa^\kappa$を$g(\alpha) = h(f(\alpha))$で定義される関数とする.
		$g(\alpha) < f(\alpha)$がすべての$\alpha \in f^{-1}(X)$で成り立つ.
		よって,$g <^* f$である.
		$f$の最小性より,ある$\gamma < \kappa$に対して $Y := \{ \alpha : g(\alpha) = \gamma\} \in U$となる.
		したがって,$D$の定義より$f(Y) \in D$であり,また,$h$は$f(Y)$上で定数$\gamma$を取る.
	\end{proof}
	
	\section{宇宙$V$の超冪と初等埋め込み}
	
	本節では,可測基数が存在すれば,内部モデルへの初等埋め込みが存在すること,逆に初等埋め込みがあれば可測基数があることを示す.また,可測基数の存在が$V=L$と両立しないことを示す.
	
	$U$を集合$S$上の超フィルターとする.$f, g \colon S \to V$に対して次の二つの関係を定める:
	\begin{align*}
		f =^* g &\iff \{ x \in S : f(x) = g(x) \} \in U, \\
		f \in^* g &\iff \{ x \in S : f(x) \in g(x) \} \in U.
	\end{align*}

	$S$を定義域とする関数全体は真クラスをなすため,同値関係$=^*$のおのおのの同値類は真クラスになってしまう.そこでScottのトリックを使って,次のように同値類のようなものを定義する.
	\[
	[f] = 	\{ g : f =^* g \land \neg (\exists h)(h = f \land \rank h < \rank g) \}
	\]
	こうすると各$[f]$は集合となる.
	$f, g \colon S \to V$に対して,$[f] \in^* [g] \iff f \in^* g$と定義する.これはwell-definedである.
	
	$\Ult = \Ult_U(V)$をすべての$[f]$ (ただし$f \colon S \to V$)全体のなすクラスとする.
	構造$\Ult = (\Ult, \in^*)$を考える.これを宇宙$V$の\textbf{超冪}という.通常のモデル理論におけるŁośの定理は宇宙の超冪でも成り立つことが確認できる:
	\[
	\Ult \models \varphi([f_1], \dots, [f_n]) \iff \{ x \in S : \varphi(f_1(x), \dots, f_n(x)) \} \in U.
	\]
	ここに$\varphi$は集合論の論理式.
	特に文を考えると,$(V, \in)$と$(\Ult, \in^*)$が初等同値なことが分かる.
	
	また,各$a \in V$に対して定数関数$c_a \colon S \to V; c_a(x) = a$を考えて,$j(a) = [c_a]$とおくと
	\[
	\Ult \models \varphi(j(a_1), \dots, j(a_n)) \iff V \models \varphi(a_1, \dots, a_n) 
	\]
	を得る.つまり,モデル理論で使っていた用語を拝借すると,$j \colon V \to \Ult$は\textbf{初等埋め込み}である.

	超冪がwell-foundedである状況を考察する.
	set-likeであることは常に成り立つ:つまり任意の$f$について
	\[
	\ext(f) = \{ [g] : g \in^* f \}
	\]
	は常に集合である.なぜなら,$g \in^* f$なる$g$を考えるとある$h =^* g$であってすべての$x \in S$で$h(x) \in f(x)$となるものをとれる.この$h$はランクが$f$以下である.よって$\rank([g]) \le \rank(f) + 1$となるので,$\ext(f)$は集合である.
	
	\begin{lem}
		$U$が$\sigma$完備な超フィルターなら,$(\Ult, \in^*)$はwell-foundedである.	
	\end{lem}
	\begin{proof}
		$\Ult$の無限$\in^*$下降列がないことを示せば良い.
		もしあったとする:$[f_0] \ni^* [f_1] \in^* \ni \dots$.
		すると各$n$について集合
		\[
			X_n := \{ x \in S : f_{n+1}(x) \in f_n(x) \}
		\]
		は$U$に属する.$U$の$\sigma$完備性より
		\[
			X = \bigcap_{n \in \omega} X_n
		\]
		も$U$に属し,特に空でない.そこから元$x \in X$を一つ取ると,
		\[
		f_0(x) \ni f_1(x) \ni f_2(x) \ni \dots
		\]
		となり,整楚性公理に反する.
	\end{proof}
	
	Mostowskiの崩壊定理は任意のwell-foundedモデルは推移的モデルと同型なことを主張しているのであった.
	よって,$U$が$\sigma$完備なら,あるクラス$M$と同型なクラス写像$\pi \colon (\Ult, \in^*) \to (M, \in)$が存在する.
	記号の乱用で$\pi([f])$のことを単に$[f]$と書く.
	合成写像$\pi \circ j$の方がもとの$j$より重要であるため,これを単に$j$と書く.したがって,初等埋め込み$j \colon V \to M$が得られる.
	
	$\alpha$が順序数ならば$j(\alpha)$も順序数であり,初等性と絶対性より$\alpha < \beta \iff j(\alpha) < j(\beta)$を得る.したがって,任意の順序数について$\alpha \le j(\alpha)$を得る.
	したがって,順序数全体のクラス$\Ordinals$は$V$と$M$の間で変わらない:$\Ordinals^V = \Ordinals^M$.すなわち,$M$は$V$の内部モデルである.
	
	初等性より$j(0) = 0$かつすべての$n \in \omega$について$j(\alpha+1) = j(\alpha)+1$であるので,すべての$n \in \omega$について$j(n) = n$である.$j(\omega) = \omega$は$\omega$の定義可能性と絶対性より分かる.
	
	\begin{defi}
		内部モデルへの初等埋め込み$j \colon V \to M$について,
		\[
		\crit(j) = \min \{ \alpha \in \Ordinals : \alpha < j(\alpha) \}
		\]
		とおき,$j$の\textbf{臨界点}と呼ぶ.
	\end{defi}
	
	\begin{lem}
		\begin{enumerate}
			\item 内部モデルへの初等埋め込み$j \colon V \to M$が非自明,すなわち$j \ne \id$のとき,臨界点$\crit(j)$は存在する.
			\item 可測基数$\kappa$とその上の$\kappa$完備非単項超フィルター$U$について$U$を使った超冪によって定まる初等埋め込み$j \colon V \to M$について,その臨界点は$\kappa$である.
		\end{enumerate}
	\end{lem}
	\begin{proof}
		(1)の証明.
		$j(x) \ne x$なるランク最小の$x$を取る.
		$y \in x$なら$\rank(y) < \rank(x)$なので,$x$のランク最小性より,$y = j(y)$を得る.よって,$y = j(y) \in j(x)$となる.したがって,$x \subset j(x)$.
		したがって,$j(x) \ne x$であることと合わせると$z \in j(x) \setminus x$がとれる.
		もし,$\rank(j(x)) = \rank(x)$なら$j(z) = z \in j(x)$となるので,初等性より$z \in x$を得て,矛盾.よって$\rank(j(x)) > \rank(x)$である.一方でランクの定義可能性と初等性と絶対性より$\rank(j(x)) = j(\rank(x))$を得るので,$j(\rank(x)) > \rank(x)$.したがって$\{ \alpha \in \Ordinals : \alpha < j(\alpha) \}$が空でないことが証明された.
		
		(2)の証明.
		$\alpha < \kappa$として$j(\alpha) = \alpha$を示す.
		$\alpha$に関する超限帰納法で示すことにすれば,任意の$\beta < \alpha$で$j(\beta) = \beta$であることを仮定して良い.
		$[f] \in j(\alpha)$を取る.すると$U$の意味でほとんどすべての$x \in S$で$f(x) < \alpha$.
		ここで$U$の$\kappa$完備性より,ある$\beta < \alpha$が存在して,ほとんどすべての$x \in S$で$f(x) = \beta$.
		よって$[f] \in j(\beta)$である.帰納法の仮定より$[f] \in j(\beta) = \beta$なので,これで$j(\alpha) = \alpha$が示された.
		
		次に$j(\kappa) > \kappa$を示す.
		対角関数$d(\alpha) = \alpha$を考える.
		$\{ \alpha : d(\alpha) < \kappa \} = S \in U$なので,$[d] < j(\kappa)$である.
		次に$\kappa \le [d]$を示す.
		$\beta < \kappa$を任意にとる.すると$\{ \alpha : \beta < d(\alpha) \} = [\beta+1, \kappa] \in U$なので,$j(\beta) < [d]$.$j(\beta) = \beta$は証明済みなので$\beta < [d]$を得る.これで$\kappa \le [d]$が示された.
		以上より,$\kappa \le [d] < j(\kappa)$である.
	\end{proof}
	
	内部モデルへの初等埋め込み$j \colon V \to M$は$j \ne \id$なら全射ではない.なぜなら,$\crit(j)$が$j$の像ではないからである.
	
	\[
	\tikz[scale=0.8]{
		\draw (0, 0) -- (0, 5) node[above] {$V$};
		\draw (3, 0) -- (3, 5) node[above] {$M$};
		
		\fill(0,2) circle (0.05) node[left] {$\kappa$};
		\fill(3,3.5) circle (0.05) node[right] {$j(\kappa)$};
		\draw[->] (0,2) -- (3,3.5);
	}
	\]
	
	\begin{thm}[Scott]
		可測基数が存在することと$V = L$は両立しない.
	\end{thm}
	\begin{proof}
		可測基数が存在し,かつ$V = L$だと仮定する.
		最小の可測基数を$\kappa$とし,$\kappa$上の非単項$\kappa$完備超フィルターを$U$とする.
		$j \colon V \to M$を$U$から生じる初等埋め込みとする.
		今, $V = L$を仮定しているので,$L$の内部モデルとしての最小性により$M = V = L$である.
		\[V \models \text{$\kappa$は最小の可測基数}\]
		と$j$の初等性により
		\[V \models \text{$j(\kappa)$は最小の可測基数}\]
		である.
		よって,$j(\kappa) = \kappa$とならないといけないが,これは$j(\kappa) > \kappa$であったことに矛盾.
	\end{proof}
	
	\begin{thm}\label{thm:elemembimpliesmeas}
		$j \colon V \to M$を非自明な初等埋め込みとする.
		このとき,$\crit(j)$は可測基数である.
		特に非自明な初等埋め込みが存在するとき可測基数が存在する.
	\end{thm}
	\begin{proof}
		$\kappa = \crit(j)$とおく.
		\[
		D = \{ X \subset \kappa : \kappa \in j(X) \}
		\]
		とおく.$D$が非単項$\kappa$完備超フィルターなことを示す.
		
		\par \textbf{主張: } $\kappa \in D$.
		\par \textbf{証明: } $\kappa < j(\kappa)$なのでよい. \hfill //
		
		\par \textbf{主張: } $\emptyset\not \in D$.
		\par \textbf{証明: } 初等性より$j(\emptyset) = \emptyset$なのでよい. \hfill //
		
		\par \textbf{主張: } $D$は共通部分で閉じている.
		\par \textbf{証明: } $X, Y \in D$とすると$\kappa \in j(X), j(Y)$.ところが初等性により$j(X \cap Y) = j(X) \cap j(Y)$なので$\kappa \in j(X \cap Y)$.よって$X \cap Y \in D$. \hfill //
		
		\par \textbf{主張: } $D$は上に閉じている.
		\par \textbf{証明: } $X \in D$かつ$X \subset Y$とする.すると初等性より$j(X) \subset j(Y)$である.したがって,$\kappa \in j(X) \subset j(Y)$を得るのでよい. \hfill //
		
		\par \textbf{主張: } $D$は超フィルターである.
		\par \textbf{証明: } $X \not \in D$とすると$\kappa \not \in j(X)$.初等性より$j(\kappa \setminus X) = j(\kappa) \setminus j(X)$となり,右辺に$\kappa$が属しているため,$\kappa \in j(\kappa \setminus X)$.つまり,$\kappa \setminus X \in D$である. \hfill //
		
		\par \textbf{主張: } $D$は非単項.
		\par \textbf{証明: } 	$\alpha \in \kappa$について$j(\{\alpha\}) = \{j(\alpha)\} = \{\alpha\}$である.第一の等式は初等性,第二の等式は臨界点$\kappa$の最小性による.この集合に$\kappa$は属さない. \hfill //
		
		\par \textbf{主張: } $D$は$\kappa$完備.
		\par \textbf{証明: } 	$\bar{X} = \seq{X_i : i < \gamma }$を$D$の元からなる列とする.ただし,$\gamma < \kappa$.
		今,初等性により$j(\bar{X}) = \seq{j(X_i) : i < j(\gamma)} = \seq{j(X_i) : i < \gamma}$である.
		したがって,再び初等性により$\bigcap_{i < \gamma} j(X_i) = j(\bigcap_{i < \gamma} X_i)$となる.
		しかし,仮定より左辺に$\kappa$が属しているため,右辺にも属する.よって,$\bigcap_{i < \gamma} X_i \in D$. \hfill //
		
		\par 以上で$D$が非単項$\kappa$完備超フィルターなことが示された.
	\end{proof}

	定理\ref{thm:elemembimpliesmeas}で作った超フィルターは正規である.
	実際,初等埋め込み$j$により$D = \{ X \subset \kappa : \kappa \in j(X) \}$と定義された超フィルター$D$が正規なことを示そう.
	$f$を$X \in D$上の押し下げ関数とすると$D$の定義より,$\kappa \in j(\{ \alpha : f(\alpha) < \alpha \})$なので,$j(f)(\kappa) < \kappa$である.そこで$\gamma = j(f)(\kappa)$とおく.
	このとき$\kappa \in j(\{ \alpha : f(\alpha) = \gamma \})$だから,再び$D$の定義より,$\{ \alpha : f(\alpha) = \gamma \} \in D$となる.よって,$D$は正規である.

	正規性は次のように超冪の言葉で特徴づけられる.

	\begin{lem}
		$D$を$\kappa$上の非単項$\kappa$完備超フィルターとする.このとき次は同値.
		\begin{enumerate}
			\item $D$は正規.
			\item $\Ult_D(V)$において $\kappa = [d]$. ここに$d$は対角関数.
			\item $D = \{ X \subset \kappa : \kappa \in j_D(X) \}$.
		\end{enumerate}
	\end{lem}
	\begin{proof}
		(1)ならば(2)の証明.$\kappa \le [d]$は明らかなので,$[d] \le \kappa$を示す.
		$f \in^* d$とすると$f$は押し下げ関数である.よって,仮定(1)よりある$\gamma < \kappa$があって,$[f] = \gamma$.
		
		(2)ならば(3)の証明.$X \subset \kappa$とする.
		\begin{align*}
			X \in D &\iff \{ \alpha < \kappa : \alpha \in X \} \in D \\
			 &\iff \{ \alpha < \kappa : d(\alpha) \in X \} \in D \\
			 &\iff [d] \in j_D(X) \text{ (Łośの定理より)}\\
			 &\iff \kappa \in j_D(X) \text{ (仮定より)}
		\end{align*}
		より良い.
		
		(3)ならば(1)の証明はこの補題の上の注意より従う.
	\end{proof}

	次に,$V$から$V$への初等埋め込みは存在しないというKunenの定理を証明する.そのために補題を用意する.
	
	\begin{lem}\label{lem:kunenslemma}
		$\lambda$を無限基数で$2^\lambda = \lambda^{\aleph_0}$なるものとする.
		このとき関数$F \colon \lambda^\omega \to \lambda$が存在して,任意の$A \in [\lambda]^\lambda$と$\gamma < \lambda$について,ある$s \in A^\omega$があって,$F(s) = \gamma$である.
	\end{lem}
	\begin{proof}
		$\seq{(A_\alpha, \gamma_\alpha) : \alpha < 2^\lambda}$を$[\lambda]^\lambda \times \lambda$の枚挙とする.
		$\alpha$に関する帰納法で,$\lambda^\omega$の元の列$\seq{s_\alpha : \alpha < 2^\lambda}$を次のように定める:$\alpha$ステージにおいて,$s_\alpha \in [A_\alpha]^\lambda$かつすべての$\beta < \alpha$について$s_\alpha \ne s_\beta$である.これは$\abs{A_\alpha^\omega} = \lambda^\omega = 2^\lambda > \abs{\beta}$より取ることができる.
		各$\alpha < 2^\lambda$について$F(s_\alpha) = \gamma_\alpha$と定める.列$\seq{s_\alpha : \alpha < 2^\lambda}$の中に現れない$s$については$F(s)$は何でもよい.
		
		この$F$が条件を満たす.実際,$A \in [\lambda]^\lambda$と$\gamma < \lambda$をとると,ある$\alpha < 2^\lambda$があって,$(A, \gamma) = (A_\alpha, \gamma_\alpha)$であり,$F(s_\alpha) = \gamma_\alpha$となる.
	\end{proof}

	\begin{thm}[Kunen]
		$j \colon V \to M$が非自明 (すなわち$j \ne id$)な初等埋め込みとしたとき,$M \ne V$である.
	\end{thm}
	\begin{proof}
		$j \colon V \to V$を非自明な初等埋め込みだとして矛盾を導く.
		$\kappa = \crit(j)$とおくと$\kappa$は可測基数.
		$\kappa_0 = \kappa, \kappa_{n+1} = j(\kappa_n)$ (for $n \in \omega$)とおくと,どの$\kappa_n$も可測基数である.
		$\lambda = \sup_{n \in \omega} \kappa_n$とおく.
		
		$j(\seq{\kappa_n : n \in \omega}) = \seq{j(\kappa_n) : n \in \omega} = \seq{\kappa_{n+1} : n \in \omega}$だから$j(\lambda) = \lambda$を得る.
		$G = \{ j(\alpha) : \alpha < \lambda \}$とおく.
		
		$\lambda$は可測基数の極限だから強極限である.さらに$\cf(\lambda) = \omega$なので
		\begin{align*}
			2^\lambda &= (2^{<\lambda})^{\cf(\lambda)} \text{ (これは一般的に成り立つ等式)} \\
			&= \lambda^{\cf(\lambda)} \text{ (強極限性)} \\
			&= \lambda^\omega		
		\end{align*}
		を得る.
		補題\ref{lem:kunenslemma}により,$F \colon \lambda^\omega \to \lambda$がとれて,すべての$A \in [\lambda]^\lambda$について$F``A^\omega = \lambda$である.
		$j$の初等性と$j(\omega) = \omega$と$j(\lambda) = \lambda$により,$j(F)$も同じ性質を持つ.
		よって,上の$G$をここでの$A$に代入すると,ある$s \in G^\omega$があって, $(jF)(s) = \kappa$である.
		
		$G$の定義より,$s$はある$t \colon \omega \to \lambda$を使って,$s(n) = j(t(n))$ (for $n\in\omega$)と表わせる.
		よって$s = j(t)$である.
		したがって,$\kappa = (jF)(s) = (jF)(jt) = j(F(t))$である.
		$\kappa$は$j$の像ではないので,これは矛盾.
	\end{proof}

	\begin{lem}
		$\kappa$を可測基数とする.
		もし$2^\kappa > \kappa^+$ならば,どんな$\kappa$上の正規$\kappa$完備非単項超フィルター$D$についても集合$\{ \alpha < \kappa : 2^\alpha > \alpha^+ \}$は$D$に属する.
		したがって,すべての基数$\alpha < \kappa$について$2^\alpha = \alpha^+$ならば,$2^\kappa = \kappa^+$である.
	\end{lem}
	\begin{proof}
		$D$を$\kappa$上の正規$\kappa$完備非単項超フィルターとし,$M = \Ult_D(V)$とおく.
		もし,$\{ \alpha < \kappa : 2^\alpha = \alpha^+ \} \in D$なら$[d] = \kappa$とŁośの定理より$2^\kappa = \kappa^+$ in $M$を得る.
		ところが,補題??より$2^\kappa = (2^\kappa)^M$かつ$\kappa^+ = \kappa$なので,$V$で$2^\kappa = \kappa^+$である.
	\end{proof}
	
	\section{ジェネリック超冪}
	
	本稿では可測基数を使わず,強制法によるジェネリックフィルターを使った超冪を考える.その応用として,Silverの定理を証明する.
	
	$\kappa$を非可算正則基数とし$I$を$\kappa$上のイデアルとする.
	$I$正値集合のなす半順序集合$(I^+, \subset)$を考える:
	\[
	I^+ = \{ X \subset \kappa : X \not \in I \}.
	\]
	$G$を$(V, P)$ジェネリックフィルターとする.
	
	以下の補題で\textbf{$M$超フィルター}というのは次を満たす$D \subset \Pow^M(\kappa)$である:
	\begin{enumerate}
		\item $\emptyset \not \in D, \kappa \in D$.
		\item $X, Y \in D$なら$X \cap Y \in D$.
		\item $X \in D$かつ$Y \in M$で$X \subset Y$ならば,$Y \in D$.
		\item $X \in M$が$X \subset \kappa$であるとき,$X \in D$または$\kappa \setminus X \in D$.
	\end{enumerate}
	
	\begin{lem}
		\begin{enumerate}
			\item $G$は$\kappa$上の$V$超フィルターで$I$の双対フィルターを拡大するものである.
			\item $V$で$I$が$\kappa$完備なら,$G$は$\kappa$完備$V$超フィルターである.
			\item $I$が正規ならば,$G$も正規である.
		\end{enumerate}
	\end{lem}
	\begin{proof}
		(1)の証明.
		$X$が$I$の双対フィルターの元ならば,$\{ Y \in I^+ : Y \subset X \}$は$I^+$の稠密部分集合なので,$X \in G$を得る.
		$V$超フィルターなことの証明はやさしい.
		
		(2)の証明.
		$\{ X_\alpha : \alpha < \gamma \}, \gamma < \kappa$を$V$に属する $\kappa$の分割とする.
		すると$\{ Y \in I^+ : Y \subset X_\alpha \text{ (for some $\alpha < \gamma$)}\}$は$I^+$の稠密部分集合である (by $I$の$\kappa$完備性).したがって,ある$X_\alpha$が$G$に属する.
		
		(3)の証明.
		$X \in G$とし$f \in V$を$X$上の押し下げ関数とする.
		すると$\{ Y \in I^+ : \text{$f$ is constant on $Y$} \}$は$X$の下で稠密である.
		よって$f$はある$Y \in G$の上で定数である.
	\end{proof}

	これから$I$は$\kappa$上の$\kappa$完備イデアルとし,全ての一点集合を含むものとする.
	すると$G$は$\kappa$上の非単項$\kappa$完備$M$超フィルターである.
	$V[G]$で超冪$\Ult_G(V)$を考える.これを\textbf{ジェネリック超冪}という.
	これは$\ZFC$のモデルだが,必ずしもwell-foundedではない.
	
	Łośの定理はジェネリック超冪でも成立する:
	\[
	\Ult_G(V) \models \varphi([f_1], \dots, [f_n]) \iff \{ \alpha \in \kappa : \varphi(f_1(\alpha), \dots, f_n(\alpha)) \} \in G.
	\]
	ここに$\varphi$は集合論の論理式で,$f_1, \dots, f_n \in V$.
	特に初等埋め込み$j_G \colon V \to \Ult_G(V); j_G(x) = [c_x]$を得る.
	
	$N = \Ult_G(V)$とする.$N$の中の順序数全体$\Ordinals^N$は線形順序付けられたクラスだが,必ずしも整列しているとは言えない.しかし,次の補題は成り立つ.
	ここで,$x \in \Ordinals^N$について$\{ y \in \Ordinals^N : y <^N x \}$が順序型$\alpha$を持つとき,記号の乱用で$x = \alpha$と書く.
	
	\begin{lem}
		\begin{enumerate}
			\item 各$\gamma < \kappa$について,$j(\gamma) = \gamma$.よって$\Ordinals^N$は順序型$\kappa$の始切片を持つ.
			\item $I$が正規ならば,$x \in \Ordinals^N$があって,$x = \kappa$である.
			実際,$[d] = \kappa$である.ただし$d$は対角関数.
			\item $j(\kappa) \ne \kappa$.
		\end{enumerate}
	\end{lem}
	\begin{proof}
		(1)の証明.
		$j \upharpoonright \gamma$が$(\gamma, \in)$と$\{ y \in \Ordinals^N : y <^N j(\gamma)\}, <^N)$の間の同型写像であることを示せばよい.
		$j \upharpoonright \gamma$の値域が$\{ y \in \Ordinals^N : y <^N j(\gamma)\}, <^N)$に含まれることは明らか.
		順序保存性,単射性は$j$の初等性より明らか.
		
		全射性を示す.$y \in \Ordinals^N$で$y <^N j(\gamma)$とする.
		$y = [f], f \in M, \dom(f) = \kappa$なる$f$を取る.
		すると$[f] <^N j(\gamma)$より
		\[
		\{ \alpha : f(\alpha) < \gamma \} \in G
		\]
		だが,左辺は$\bigcup_{\beta < \gamma} \{\alpha : f(\alpha) = \beta \}$と書けるため,$G$の$\kappa$完備性により,ある$\beta < \kappa$について $\{\alpha : f(\alpha) = \beta\} \in G$である.よって,$y = [f] = j(\beta)$.
		
		(2)の証明.
		$j \upharpoonright \kappa$が$\kappa$と$\{ y \in \Ordinals^N : y <^N [d] \}$の間の同型となることを示す.
		$j \upharpoonright \kappa$の値域が$\{ y \in \Ordinals^N : y <^N [d] \}$に収まることは,各$\alpha \in \kappa$について$\seq{\alpha, \alpha, \alpha, \dots} \in^* \seq{0, 1, 2, \dots}$よりよい.
		順序保存性,単射性は再び明らかである.
		
		全射性を示す.$[f] \in \Ordinals^N$で$[f] <^N [d]$なるものをとる.
		すると$f$はある$G$のメンバーの上で押し下げ関数である.
		$G$が正規なので,ある集合$X \in G$上で$f$は定数関数である.
		その定数$\alpha < \kappa$について$j(\alpha) = [f]$を得る.
		
		(3)の証明.(2)の証明は全射性以外,正規性を使っていない.そこで$\range(j \restrict \kappa) \subset \{ y \in \Ordinals^N : y <^N [d] \}$は順序型$\kappa$を持つ.
		よって,$\{ y \in \Ordinals^N : y \le^N [d] \}$は順序型$\kappa + 1$の部分集合を持つ.
		$[d] < j(\kappa)$であるため,$\{ y \in \Ordinals^N : y <^N j(\kappa) \}$も順序型$\kappa + 1$の部分集合を持つ.
		よって,この集合は順序型$\kappa$を持つことはない.
	\end{proof}
	
	\begin{thm}[Silver]
		$\kappa$を特異基数で$\cf(\kappa) = \omega_1$とする.また,すべての$\lambda < \kappa$で$2^\lambda = \lambda^+$と仮定する.このとき$2^\kappa = \kappa^+$.
	\end{thm}
	\begin{proof}
		$(\statone, \subset)$を$\omega_1$の定常集合全体が包含関係で作る半順序集合とする.
		$G$を$(V, \statone)$ジェネリックフィルターとする.
		$V[G]$で議論する.
		$G$は$\omega_1^M$上の正規$\sigma$完備$M$超フィルターである.
		$(N, \mathrel{\varepsilon}^N) = \Ult_G(V)$をジェネリック超冪とし,$j \colon V \to N$を誘導される初等埋め込みとする.
		
		$\seq{\kappa_\alpha : \alpha < \omega_1}$を$V$の中で単調増加連続な基数の列で$\kappa$に収束するものとする.
		$e$を$N$の中の基数とし,$e(\alpha) = \kappa_\alpha$で定められる関数によって表現されるものとする.$e^+$を$N$の中での$e$の後続基数とする.
		
		$x \in N$に対して$\ext(x) = \{ y \in N : y \mathrel{\varepsilon}^N x \}$とおく.これは$V[G]$の集合である.
		この定義より特に
		\[
		\ext(\Pow^N(e)) = \{ x \in N : N \models ``x \subset e" \}
		\]
		である.
		
		\par \textbf{主張A: } $\abs{\Pow^V(\kappa)} \le \abs{\ext(\Pow^N(e))}$.
		\par \textbf{証明: } $V$の中の$X \subset \kappa$について関数$f_X$を$f_X(\alpha) = X \cap \kappa_\alpha$ ($\alpha \in \omega_1$)と定める.$f_X$が表現する$N$の元は,$N$の中で$e$の部分集合である.
		また,$X \ne Y$なら,関数$f_X$と$f_Y$はゆくゆく異なるので,異なる$N$の元を表現する. \hfill //
		
		\par \textbf{主張B: } $\abs{\ext(\Pow^N(e))} = \abs{\ext(e^+)}$.
		\par \textbf{証明: } $V$で任意の$\alpha$について$2^{\kappa_\alpha} = \kappa_\alpha^+$であることから,Łośの定理より,$N$で$2^e = e^+$が成り立つ.つまり$F \in N$がとれて,$N \models F \colon 2^e \to e^+ \text{ 全単射}$となる.
		各$x \in \ext(\Pow^N(e))$について$y \in N$で$N \models y = F(x)$となる元を割り当てる関数を$\tilde{F} \colon \ext(\Pow^N(e)) \to \ext(e^+)$とする.これは全単射であることが確認できるので,主張が示された. \hfill //
		
		\par \textbf{主張C: } 任意の$a \mathrel{\varepsilon}^N e$について,$\gamma < \omega_1^V$が存在して,$a \mathrel{\varepsilon}^N j(\kappa_\gamma)$である.
		\par \textbf{証明: } $a \mathrel{\varepsilon}^N e$を任意にとり,関数$f$が$a$を表現するとする.
		このときある$X \in G$があって,全ての$\alpha \in X$で$f(\alpha) < \kappa_\alpha$である.
		ここで極限順序数全体の集合はclubなので$G$に属する.よって,上で取った$X$は全ての元が極限順序数だと仮定して良い.
		したがって,列$\seq{\kappa_\alpha : \alpha < \omega_1}$を連続で取っていたことから,$f(\alpha) < \kappa_{\gamma(\alpha)}$が,ある$\gamma(\alpha) < \alpha$について成り立つ.
		$\gamma$は押し下げ関数だから,ある$\gamma < \omega_1^V$が存在して,ある$Y \in G$について,任意の$\alpha \in Y$で$f(\alpha) < \kappa_\gamma$となる.つまり$a \mathrel{\varepsilon}^N j(\kappa_\gamma)$を得る. \hfill //
		
		\par \textbf{主張D: } $\abs{\ext(e)} \le \kappa$.
		\par \textbf{証明: } 各$\gamma < \omega_1^V$について,$\abs{j(\kappa_\gamma)} \le \abs{(\kappa_\gamma^{\aleph_1})^V} < \kappa$である.第一の不等号は$j(\kappa_\gamma)$の元というのはつねに$\kappa_\gamma$の元を値に取る$\omega_1$列で表現されるからである.よって,主張Cと合わせて,$\abs{\ext(e)} \le \kappa$を得る.  \hfill //
		
		\par \textbf{主張E: } $\abs{\ext(e^+)} \le \kappa^+$.
		\par \textbf{証明: } もし,$x \mathrel{\varepsilon}^N e^+$なら,$N$の中に$x$から$e$への単射があるから,主張Bと同じ方法によって,$\ext(x)$から$\ext(e)$への単射を得る.したがって,$\ext(e^+)$は全順序集合で,どの始切片もサイズたかだか$\kappa$を持つので,$\ext(e^+) \le \kappa^+$を得る (\cite{jech2006set}のExercise 5.3を参照).	  \hfill //
		
		主張A, B, Eを組み合わせると
		\[
			\abs{\Pow^V(\kappa)} \le \abs{\ext(\Pow^N(e))} \le \abs{\ext(e^+)} \le \kappa^+
		\]
		を得る.これは$V[G]$での不等式である.
		ところが,$\abs{P} = 2^{\aleph_1} < \kappa$であるため,chain conditionにより,$V$の全ての$\kappa$以上の基数は$V[G]$でも基数である.よって
		\[
		\abs{\Pow^V(\kappa)}^V \le (\kappa^+)^V
		\]
		を得る.これが欲しかった結論である.
	 \end{proof}
	
	\section{峻厳イデアル}
	
	\nocite{*}
	\printbibliography[title={参考文献}]

	
\end{document}
